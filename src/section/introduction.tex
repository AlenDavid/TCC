\section{Complexity and Refactoring}

The first step of this article is to explain what is Complexity and how to observe it.

In Software engineering, Code Complexity can be expressed by how much cognitive effort one must provide to understand
and maintain a source code \cite{article:fmricc}. This creates a barrier that in most cases hide bugs and problems.
To measure complexity, one could start by counting the lines of code they target system have. Lines of code is usually
referenced when discussing complexity \cite{article:rustcc}. In this paper, we count lines of code, physical lines (instructions),
logical lines (statements), comment lines and blank lines.
In literature, its common to see the use of Halstead measurements to describe complexity \cite{article:complexity_with_halstead}.
Maurice Halstead, in 1977, created a set of rules that calculates the effort to understand target software, number of bugs
expected in the program, average time that is taken to implement the program. To do so, it first measures the distinct and total
operators and operands the program has. Halstead understands software as a sequence of operands and operators, and uses
this information to calculate:

N = The length of a program (the total occurrences of operands and operators);

n = The vocabulary (the sum of distinct operands and operators);


A common model used by companies to describe Complexity is McCabe's Cyclomatic Complexity and Sonar's Cognitive Complexity
\cite{article:sonarpaper}.

We will also use number of arguments, number of exits and maintainability metrics in our paper.


When the effort to make a significant change increase, \cite{book:refactoring} will recommend one engineer to practice
refactoring. Refactor is a conjunction of defined techniques tool be used as leverage to reduce the amount of time spent
writing code that makes impactful changes in a system.

The goal of this paper is to quantify the impact of those said techniques, not to say when and where to apply them.
To achieve this goal, we will use the \cite{article:mozilla} tool to analyze each refactor technique listed on
\cite{book:refactoring} book. This tool offers a handful set of metrics researched by
different writers mentioned in the beginning of this chapter. They will be separated in four major classes: code size,
represented by lines of code; vocabulary size, by Halstead metrics; control-flow complexity, McCabe's Cyclomatic
Complexity; and data-flow complexity.

