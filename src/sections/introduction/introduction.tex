\section{Code Complexity and Refactoring}

The first step of this article is to explain what is Code Complexity and how to observe it.

In Software engineering, Code Complexity can be expressed by how much cognitive effort one must provide to understand
and maintain a source code \cite{article:fmricc}. This creates a barrier that in most cases hide bugs and problems.
To measure code complexity, one could start by counting the lines of code they target system have. Lines of code is usually
referenced when discussing code complexity \cite{article:rustcc}. In this paper, we count lines of code, physical lines (instructions),
logical lines (statements), comment lines and blank lines.


When the effort to make a significant change increase, \cite{book:refactoring} will recommend one engineer to practice
refactoring. Refactor is a conjunction of defined techniques tool be used as leverage to reduce the amount of time spent
writing code that makes impactful changes in a system.

The goal of this paper is to quantify the impact of those said techniques, and not to recommend when and where to apply them.
To achieve this goal, we will use the \cite{article:mozilla} tool to analyze each refactor technique listed on
\cite{book:refactoring} book. This tool offers a handful set of metrics researched by
different writers mentioned in the beginning of this chapter. These metrics are separated in four major classes: code size,
represented by lines of code; vocabulary size, by Halstead metrics; control-flow complexity, McCabe's Cyclomatic
Complexity; data-flow complexity.
