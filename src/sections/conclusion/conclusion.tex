\section{Conclusion}

In conclusion, the analysis of the code changes reveals a trend toward increased complexity and size, which could impact
long-term maintainability. While additional functionality and structure have been added, the growth in Cyclomatic
Complexity and Halstead Metrics indicates that the code is becoming more intricate and harder to understand, requiring
more effort for maintenance. Similarly, the increase in Lines of Code shows a larger codebase, which may lead to higher
potential for errors and difficulties in upkeep.

Although the Maintainability Index has decreased, reflecting the higher complexity, the rise in the Number of Methods
suggests that the code has been modularized to some extent. However, this modularity could be contributing to the higher
number of methods and functions, potentially increasing the burden on developers to manage and update the code.

To improve maintainability, it is important to focus on simplifying the code where possible, reducing unnecessary
complexity, and maintaining a balance between adding features and keeping the system manageable. This could involve
refactoring, optimizing code structure, and enhancing documentation to ensure that future changes are easier to
implement without compromising the system's sustainability.
